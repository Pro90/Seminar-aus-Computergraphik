\section{Overview of Tangible User Interfaces}

In this section, we give an overview over Tangible User Interfaces and their applications. We will present several TUIs that we regard as extraordinary and explain the techniques behind them, as well as their area of application. Most of the presented TUIs are aimed at collaborative tasks.

\subsection{The reacTable}

The reactAble, presented by \cite{jorda07}, is a musical instrument based on a tabletop TUI. Fiducial Markers represent musical objects, which generate sound according to their relation to each other. The markers are tracked by an IR camera. According to their attached symbol, each object has a dedicated function. The objects can be categorized in six different functional groups: audio generators, audio filters, controllers, control filters, mixers and global objects. \shortcite{jorda07}

ReacTIVision, the computer vision system behind reactAble, tracks the fiducial markers and sends the output data to an audio synthesizer. The waveforms generated by the synthesizer, as well as the data from the ReacTIVision tracker are sent to a visual synthesizer. The visual synthesizer projects visual feedback back onto the tabletop. The audio lines that connect objects show the real resulting waveforms. Visual feedback is also used to monitor the objects state and internal parameters. Fingers can be used to either modify the objects parameters, or to cut (i.e. mute) audio connections between objects. \shortcite{jorda07}

Modular synthesis is used for the sound generation process. Modular synthesis is based on the interconnection of sound generators and sound processor units. In reactAble, automatic connections between objects are made depending on the type of objects involved and the proximity between them. By moving objects around and bringing them into relation to each other, performers construct and play instruments at the same time. \shortcite{jorda07}

