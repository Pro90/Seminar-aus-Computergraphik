\section{Typical designs of Tangible User Interfaces}

\subsection{Tangible User Interfaces as a complement to graphical user interfaces}

The graphical user interface (GUI) with the input devices of mouse and keyboard falls short in embracing the rich interface modalities between people and the physical environment they inhabit \cite{ullmer97}. 
One of the attemps in broadening the input possibilities to different devices is the metaDESK system introduced by Ullmer and Ishii \cite{ullmer97}. They describe a "Tangible User Interface" (TUI) as a user interface employing physical objects, instruments, surfaces, and spaces as physical interfaces to digital information. The metaDESK system consists of: a desk, a nearly-horizontal backprojected graphical surface; an active lens an arm-mounted flat-panel display; one or more passive lenses, an optically transparent surface through with the desk projects; and an assortment of physical objects an instruments which are used on the desk's surface. The components are sensed by an array of optical, mechanical and electromagnetic field sensors. The focus lies on the use of real physical objects as driving elements of human-computer interaction. Their approach allthoug tries to take elements of the GUI and bringing it into the real world as well as pushing forward from the unaugmented physical world, inheriting from vaious historical instruments and devices often "obsoleted" by the advent of the computer, like the active lens which is based on a jeweler's magnifying lens. The models for the objects are taken from everyday objects from home, scientific instruments or drawing and design tools. The material they used was transparent machined acrylic, designed to minimize occlusion of the desk surface.
The GUI icons are instantiated as "phicons" (physical icons), menus and handles are instantiated as TUI "trays" and "phandles" (physical handles), scales and scrollbars as TUI instruments such as a rotation constraint instrument. To test the system they implemented a prototype application called "Tangible Geospace" allowing interaction with geographical space.
The models themselves act as information containers about the object they represent as well as physical handles for manipulating the map.
The arm mounted active lens is coupled to the models and displays three-dimensional views of the scene and moving the lens makes it possible to navigate throu 3D space. This allows a seamless interaction with three spaces at once: the physical space of the object, the 2D graphical space of the desk's surface and the 3D graphical space of the active lens.
It is also possible to place a second object on the table, allowing the user to scale or rotate the map by moving the objects with respect to each other. This also allows collaboration as each object may be manipulated by an individual user. The sensing is performed by a computer-vision system inside the desk unit, along with magnetic -field position sensors and electrical contact sensors.
The passive lenses consist of a transparent surface that functions as an independent display when augmented by the back-projected desk. Since they are passive transparent surfaces, many vaiously afforded lenses might be used simultaneously with no additional active display resources.
As alternative to the two phicon scaling/rotation interaction, a rotation constraint instrument made of two cylinders mechanically coupled by a sliding bar might be used.
Albeit the extension of input methods it is not the goal of metaDESK to replace GUIs, but rather to complement them by providing new opportunities for human-computer interaction. 

\subsection{Including social aspects}

Hornecker and Buur \cite{hornecker06} extend the thought of tangible user interfaces further to 'tangible interactions'.


% Design and Hardware