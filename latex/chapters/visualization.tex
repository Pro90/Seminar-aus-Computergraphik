\section{Tangible User Interfaces in Visualization}
The purpose of this chapter is to introduce Tangible User Interfaces which are settled in the domain of Visualization. We will give an overview of TUIs in Visualization we consider noteworthy. In the previous chapter, some of the presented TUIs could also be labeled as TUIs for Visualization, because some output is visualized by the system. In the reacTable for example, the sound waves between different musical objects are visualized on the tabletop. However, the reactAble is designed as a musical instrument. In this chapter, we will focus explicitly on TUIs, which sole purpose is the Visualization of data.

\section{Tangible Views for Information Visualization}
Tangible Views is a Tangible User Interface for Information Visualization presented by \cite{spindler10}. It consists of several handheld displays, which allow to interact with the visualized data in a more direct way. Similar to Paper Windows, a TUI presented by \cite{holman05}, the information is project onto cardboard displays (tangible views) as well as a tabletop. The setup also consists of several IR cameras, which track the tangible views and make them spatially aware. Gestures performed on the tangible view are recognized by the system as well. 
