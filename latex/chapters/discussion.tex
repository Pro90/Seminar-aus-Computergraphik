\section{Discussion}
In the previous section, we have presented several Tangible User Interfaces in the domain of Visualization. Each of these TUIs follows a different approach in visualizing data. Furthermore, the presented are all settled in different areas of Visualization. This ranges from Information Visualization in Tangible Views to Visualization of atoms and molecules in Augmented Chemistry. In this section, we will evaluate and discuss advantages and drawbacks of the different systems. 

\subsection{Portability}
Most of the TUIs rely on tabletop environments to display visualized content. This systems may be intuitive in terms of interaction with the displayed content, but they don't seem to be very portable. To install and use such a system, large tabletops have to be constructed and a projector has to be installed under the tabletop or on the ceiling. All TUIs except the Augmented Chemistry system and the G-nome surfer rely on this environment. The G-nome surfer is implemented on a Microsoft surface, a big multi-touch screen. It seems to be very portable for a multi-touch desk, but interactions like in Tangible Views are not possible. Augmented Chemistry also needs a projector to function correctly. However, the rear-display of the system seems not so large as the tabletop displays in the other TUIs, so the system could be moved easily. 

\subsection{Visualization Techniques}
In Tangible Views, users use cardboard displays to interact with the system. These displays seem to be very handy and lightweight \shortcite{spindler10}. Because the Tangible Views are tracked in 3D-space, many interactions techniques seem to be possible. Interactions can also take place on the displays themselves, by using a pen or with multi-touch gestures. Tangible Views seems to be very adoptable concerning different types of Visualizations. Searching through visualized information appears to be intuitive and natural. 

The steerable tangible interface presented by \cite{lee09} seems to be very specialized. Only certain styles of Visualization can be applied effectively. The steerable ring of the TUI appears to be intuitive in terms of usage, but only Visualizations with different layers on top of each other can be explored with the TUI. Furthermore, the user has to be attention not to cover the LEDs located in the ring, because otherwise the system can not track the ring and will stop working. 

The G-Nome surfer presented by \cite{shaer10} consists of a large multi-touch display. Users who are already know touch surfaces from devices like smartphones and tablets, will quickly get familiar with the TUI. One advantage of G-nome surfer is that is does not rely on projectors for displaying information. Users working on the environment cannot occlude fiducial markers for tracking purposes or projected content. 

Augmented Chemistry \cite{fjeld02} is different from the other interaction techniques: It uses augmented reality to visualize information. Fiducial markers are tracked by a camera, which is facing the user. The occlusion of these markers could definitely be a problem. Furthermore, many tools are used for the interaction with the system. Tools have to be used simultaneously. Users could be overwhelmed by the tools and their interactions. To avoid this problem, detailed instructions have to be given to the users. 

