\section{Discussion}
In the previous section, we have presented several Tangible User Interfaces in the domain of Visualization. Each of these TUIs follows a different approach in visualizing data. Furthermore, the presented are all settled in different areas of Visualization. This ranges from Information Visualization in Tangible Views to Visualization of atoms and molecules in Augmented Chemistry. In this section, we will evaluate and discuss advantages and drawbacks of the different systems. 

\subsection{Portability}
Most of the TUIs rely on tabletop environments to display visualized content. This systems may be intuitive in terms of interaction with the displayed content, but they don't seem to be very portable. To install and use such a system, large tabletops have to be constructed and a projector has to be installed under the tabletop or on the ceiling. All TUIs except the Augmented Chemistry system and the G-nome surfer rely on this environment. The G-nome surfer is implemented on a Microsoft surface, a big multi-touch screen. It does seem portable , 