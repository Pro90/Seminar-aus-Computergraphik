\section{Discussion}
In the previous section, we have presented several Tangible User Interfaces in the domain of visualization. Each of these TUIs follows a different approach in visualizing data. Furthermore, the presented TUIs are all settled in different areas of visualization. This ranges from information visualization in Tangible Views to the visualization of atoms and molecules in Augmented Chemistry. In this section, we will evaluate and discuss different aspects of the presented TUIs.

\subsection{Portability}
Most of the TUIs rely on table-top environments to display visualized content. This systems may be intuitive in terms of interaction with the displayed content, but they don't seem to be very portable. To install and use such a system, large table-tops have to be constructed and a projector has to be installed under the table-top or on the ceiling. All TUIs except Augmented Chemistry, Portico and the G-nome surfer rely on this environment. The G-nome surfer is implemented on a Microsoft surface tablet, a big multi-touch screen. It seems to be very portable for a multi-touch desk, but three-dimensional interactions like in Tangible Views are not possible. Augmented Chemistry also needs a projector to function correctly. However, the rear-display of the system seems not so large as the table-top displays in the other TUIs, so the system could be moved easily. Portico seems to be a cheap and portable way to enhance tablet PCs for tangible interaction. 

\subsection{Occlusions}
Many of the presented TUIs rely on marker tracking and top projecting facilities. Users could occlude markers of the tracking system, so the system would get no or wrong tracking results. As a consequence, the TUI will produce inaccurate results or entirely stop working. The user has to take care that tracking markers are not obscured. The markers should also be placed cleverly in the design step of the system, to avoid occlusions in the first place. 

By projecting content from the ceiling, hands or interaction tools could be beneath the projector and the table-top. Information could therefore be lost or displayed distorted. Paper Windows is a TUI were these kind of occlusions could happen often, because information is not only projected onto the Paper Windows, but the Paper Windows are also used as an interaction tool. The user can perform gestures directly on the paper and occlude the information projected from above. Tabletop environments, where the IR camera and the projector are seated under the table, avoid occlusions by tracking and projecting information from below. 

\subsection{Visualization Techniques}
reacTable visualizes sound waves between objects placed onto the table-top. This helps the artists to distinguish between the different sounds of the objects. The object states are also projected onto the table-top. By moving or rotating objects, their states can be changed.

Urban planning workbenches can visualize different types of simulations. The user has to know different types of tools and their application to interact with the simulation correctly. Some tools have to be placed into the three-dimensional scene. When a lot of models are laid out in the scene, it could be hard to place the tools where the user wants them. 

In Tangible Views, users use cardboard displays to interact with the system. These displays seem to be very handy and lightweight. Because the Tangible Views are tracked in 3D-space, many interactions techniques seem to be possible. Interactions can also take place on the displays themselves, by using pens or multi-touch gestures. Tangible Views seems to be very adoptable concerning different types of visualizations. Searching through visualized information appears to be intuitive and natural. 

The steerable tangible interface presented by \cite{lee09} seems to be very specialized. Only certain styles of visualizations can be applied effectively. The steerable ring of the TUI appears to be intuitive in terms of usage, but only visualizations with different layers on top of each other can be explored with the TUI. 

The G-Nome surfer consists of a large multi-touch display. Users who already know touch surfaces from devices like smartphones and tablets, will quickly get familiar with the TUI. One advantage of G-nome surfer is that it does not rely on projectors for displaying information. Users working on the environment cannot occlude fiducial markers for tracking purposes or projected content. 

Augmented Chemistry is different from the other interaction techniques: It uses augmented reality to visualize information. Fiducial markers are tracked by a camera, which is facing the user. Many tools are used for interacting with the system. Tools have to be used simultaneously. Users could be overwhelmed by the tools and their interactions. To avoid this problem, detailed instructions have to be given to the user. 