\section{Introduction}

%% The ``\copyrightspace'' command must be the first command after the 
%% start of the first section of the body of your paper. It ensures the
%% copyright space is left at the bottom of the first column on the first
%% page of your paper.

\copyrightspace


The graphical user interface (GUI) with the input devices of mouse and keyboard falls short in embracing the rich interface modalities between people and the physical environment they inhabit \cite{ullmer97}. Therefore attempts have been made to interweave the virtual world with the physical world. These attempts are reflected in the achievements of Tangible User Interfaces (TUI). 

One of the main goals in using Tangible User Interfaces is to combine visualization of data with direct interaction. 
In common user interfaces the interaction is limited to indirect input methods such as mouse and keyboard. However, indirect pointing devices fail to utilize the powerful capabilities of the human motor system. Therefore researchers attempt to find a way of interaction where the user can directly touch and manipulate the objects of interest.

%In this paper we first give a general overview on how TUIs work and what the main challenges are. As well as give some examples of recent works.

In the first section we will give a general overview on TUI application areas and design aspects.
We describe different approaches in implementing Tangible User Interfaces and take a look at how a system setup could look like depending on the specific area it is designed to work in. 

Therefore we take a look at the following examples: the metaDESK by Ullmer and Ishii \cite{ullmer97}, Tangible Views by Spindler, Tominski, Schuhmann and Dachselt \cite{spindler10}, Interactive Textbook by Koike, Sato, Kobayashi Y., Tobita and Kobayashi M. \cite{koike00}, Tangible Query Interfaces by Ullmer, Ishii and Jacob \cite{ullmer03}, the reacTable by Jord\`{a}, Geiger, Alonso and Kaltenbrunner \cite{jorda07} and Urp by Underkoffler and Ishii \cite{underkoffler99}.

Next is a short section with a discussion of the work of Hornecker and Buur \cite{hornecker06} who strive to integrate social aspects into tangible user interfaces. 

In the next section we will give some examples of TUIs and describe how some of the challenges can be solved. 
This section is divided into three parts: 
\begin{itemize}
\item Table-top environments - namely the reacTable, used for the visualization of music, and the TARboard, a table-top game environment.
\item Urban Planing Workbenches - namely Urp, designed to simulate different environmental influences like shadows or wind effects, and the Luminous Table, which extends the functionality of Urb even further.
\item Other forms of Tangible User Interfaces - like Portico, enabling tangible interaction on and around tablet computers, Paper Windows, simulating digital paper by projecting digital content onto physical paper and �D Tractus a three-dimensional user interface to monitor and control a team of independent robots.
\end{itemize}

Section five focuses on Tangible User Interfaces in Visualization. It examines the work of \cite{spindler10} and describes interaction patterns and the practical use of the system, the Steerable Tangible Interface for Multi-Layered Contents by \cite{lee09} in regards of configuration and application, the G-nome Surfer by \cite{shaer10}, a multi-user, multi- touch table for collaborative exploration of genomic data and Augmented Chemistry - A Tangible User Interface for Chemistry Education by \cite{fjeld02}, a TUI for working and interacting with molecular models.

This is followed by a discussion in which we evaluate the beforementioned systems.
