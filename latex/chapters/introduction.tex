\section{Introduction}

%% The ``\copyrightspace'' command must be the first command after the 
%% start of the first section of the body of your paper. It ensures the
%% copyright space is left at the bottom of the first column on the first
%% page of your paper.

\copyrightspace


The graphical user interface (GUI) with the input devices of mouse and keyboard falls short in embracing the rich interface modalities between people and the physical environment they inhabit \cite{ullmer97}. Therefore attempts habe been made to interweave the virtual world with the physical world. These attempts are reflected in the achievements of Tangible User Interfaces (TUI). 

One of the main goals in using Tangible User Interfaces is to combine visualization of data with direct interaction. 
In common user interfaces the ineraction is limited to indirect input methods such as mouse and keyboard. However, indirect pointing devices fail to utilize the powerful capabilities of the human motor system. Therefore researchers attempt to find way of interaction where the user can directly touch and manipulate the objects of interest.

%In this paper we first give a general overview on how TUIs work and what the main challenges are. As well as give some examples of recent works.

In the first section we give a general overview on how TUIs work and what the main challenges are.
Furthermore wel regard different design spaces an application areas. 
We describe different approaches in implementing Tangible User Interfaces and take a look at how a system setup could look like. 
Therefore we take a look at the metaDESK system by Ullmer and Ishii \cite{ullmer97}, one of the early advances in TUIs. 
Then wel describe a more lightweight interaction system called 'Tangible Views' by Spindler et al. \cite{spindler10}.
We conclude the section with a discussion of the work of Hornecker and Buur \cite{hornecker06} who strive to integrate social aspects into tangible user interfaces. 

In the next section wel give some examples of TUIs and describe how some of the challenges can be solved. 
This section is divided into three parts: 
\begin{itemize}
\item Table-top environments - namely the reacTable, used for the visualization of music, and the TARboard, a table-top game environment.
\item Urban Planing Workbenches - namely Urp, designed to simulate different environmental influences like shadows or wind effects, and the Luminous Table, which extends the functionality of Urb even further.
\item Other forms of Tangible User Interfaces - like Portico, enabling tangible interaction on and around tablet computers, and Paper Windows, simulating digital paper by projecting digital content onto physical paper.
\end{itemize}

The next section then focuses on Tangible User Interfaces in Visualization.

This is followed by a section about Tangible Views for Information Visualization.

The paper is concluded with a discussion.
