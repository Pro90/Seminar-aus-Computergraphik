\section{Conclusion}
We have examined different designs of Tangible User Interfaces in this work. Furthermore, we have presented and discussed several examples TUIs. Although the focus of this work lies in Tangible User Interfaces for VIsualization, we have also presented TUIs for other domains. A discussion about the pros and cons of the different TUIs followed their presentation. 

TUIs can be intuitive tools for the examination of visualized data. Instead of using the classic mouse and keyboard interaction, they enable users a more natural way of exploration and interaction with digital content. By using graspable objects and multi-touch gestures, users learn the interaction patterns more easily. This results in a faster and more comfortable way of examining visualized information. The TUI interaction paradigms work especially well for huge amount of data with multiple layers. However, TUIs are normally specialized on certain Visualization tasks. Therefore, TUIs can only interact with certain styles of Visualizations effectively. 